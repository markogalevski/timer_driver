After reading Jacob Beningo\textquotesingle{}s book, Reusable Firmware Development, I\textquotesingle{}ve decided to begin the arduous process of building my own easily portable H\+AL to use for future projects. It feels as if all I\textquotesingle{}ve been developing for ages at this point is drivers.

The general principle is as follows\+:
\begin{DoxyItemize}
\item A general \hyperlink{gpio__interface_8h}{gpio\+\_\+interface.\+h} defines the api which will be exposed to applications. It will be this file which is included by the application. It is designed in a way to be 100\% non-\/platform dependent. Changes required may be the modification of uint32\+\_\+t types to uint16\+\_\+t to match an architecture.
\item The micrcontroller specific gpio\+\_\+stm32f4xx.\+c file contains an M\+CU specific implementation of the peripheral. Accompanying it are a config .c/.h pair. These define a table of init structures for each instance of the peripheral, as well as all relevant typedefs. These files will need to be changed to port the driver. Time to port a gpio driver seems to be less than a day\textquotesingle{}s worth of work.
\item To port the driver\+: simply prepare the M\+CU specific c and config files and set \hyperlink{gpio__interface_8h}{gpio\+\_\+interface.\+h} to include the appropriate xxxxx\+\_\+config.\+h file, and exchange the source files. 
\end{DoxyItemize}